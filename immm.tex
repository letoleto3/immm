\documentclass[12pt]{extarticle}
\usepackage[utf8]{inputenc}
\usepackage[english,russian]{babel}
\usepackage{vmargin}
\usepackage{indentfirst}
\usepackage[T2A]{fontenc}
\usepackage{graphics}
\usepackage{amsthm}
\usepackage{amsbsy}
\usepackage{amsmath}
\usepackage{amssymb}
\usepackage{amsfonts}
\usepackage{mathtext}
\usepackage[pdftex,a4paper,colorlinks,linkcolor=blue,citecolor=blue]{hyperref}	

\usepackage{mathtext}
\usepackage{mathenv}
\usepackage[pdftex]{graphicx}
\usepackage{array}
\usepackage{graphicx,xcolor}
\usepackage{xcolor}
\usepackage{float}
\usepackage{longtable}

\usepackage{tikz}
\usepackage{verbatim}

\usepackage{hyperref}
\usepackage{cite,enumerate,float,indentfirst}
\linespread{1.5}

\usepackage{amsthm}
\usepackage{amsbsy}
\usepackage{amsmath}
\usepackage{amssymb}
\usepackage{amsfonts}
\usepackage{mathtext}
\usepackage{float}

\usepackage{array}
\usepackage{longtable}


\newtheorem{example}{Пример}[section]
\newtheorem{definition}{Определение}[section]
\newtheorem{proposition}{Предложение}
\newtheorem{theorem}{Теорема}[section]
\newtheorem{corollary}{Следствие}[section]
\newtheorem{lemma}{Лемма}

\DeclareMathOperator{\tr}{tr}

\newcommand{\dif}[2]{\frac{\partial #1}{\partial #2}}
\DeclareMathOperator{\Log}{Log}
\begin{document}

\renewcommand{\contentsname}{Содержание}
\newenvironment{changemargin}[1]{
  \begin{list}{}{
    \setlength{\voffset}{#1}
  }
  \item[]}{\end{list}}


\begin{titlepage}
	\begin{center}
		\textsc{ 
		{
			Московский Государственный Университет\\ им. М.\,В.~Ломоносова \\\vspace{15pt}
 			Механико-математический факультет \\ 
 			}
 		}  
 		\large{
 		\vspace{10em}
		\textsc{\textbf{\large{Аннотация к книге}}\\ \vspace{15pt}
			Якоба Бенулли \\ 
			\Large{
				\textbf{<<О законе больших чисел>>} \\\vspace{10pt}
			} 
		}
		}
	\end{center}
	\vspace{5em} 
	\newbox{\lbox} 
	\newlength{\maxl} \setlength{\maxl}{\wd\lbox} \hfill\parbox{15em}
	{
		\hspace*{7em}\hspace*{-7em}
		\textbf{Выполнила}\\
	 	\hfill\hbox to\maxl{\,\,Я.\,В.~Нагорных, гр.\,410} \\ 
	}
	
	\newbox{\lbox} \hfill\parbox{15em}
	{
		\hspace*{7em}\hspace*{-7em}
		\textbf{Преподаватели}\\
		\hfill\hbox to\maxl{\,\,к.\,ф.-м.\,н. М.\,А.~Подколзина}\\
		\hfill\hbox to\maxl{\,\,д.\,ф.-м.\,н. С.\,С.~Демидов}\\
	}
	\\ \vspace{8em}
	\textsc{
	\begin{center} 
		Москва \\ 
		2018
	\end{center} }
\end{titlepage}


\tableofcontents

\newpage
\section{Краткая биография}
Якоб Бернулли родился 6 января 1655 года в швейцарском Базеле.
Младший брат Якоба Иоганн также стал математиком.
Его отец Николай был преуспевающим фармацевтом. 

По его же желанию Якоб пошел учиться богословию в Базельский университет, однако увлёкся математикой, и в последствии изучил ее самостоятельно. 
В университете Якоб выучил французский, итальянский, английский, латинский и греческий языки.
В 1671 году он получил учёную степень магистра философии.
Позднее Бернулли много путешествовал по Европе, что дало ему знакомства с Гюйгенсом, Гуком и Бойлем.

В 1683 году вернулся в Базель после путешествий и вскоре женился на Юдит Штупанус. Позднее у них родились сын и дочь.

В 1683 году начал читать лекции по физике в Базельском университете. 
С 1687 года избран профессором математики в этом университете и работал там до конца жизни.
В то же время он начал изучать труды Лейбница по анализу и с энтузиазмом начал освоение нового исчисления. 
Также он привлек в изучение анализа своего младшего брата Иоганна.
Позднее Лейбниц откликнулся на письмо Якоба, и их взаимная переписка обогатила зарождающийся анализ.

В 1699 году Якоб Бернулли стал членом Парижской академии наук, а в 1702 году -- Берлинской академии наук.

Первое триумфальное выступление молодого математика относится к 1690 году. Якоб решает задачу Лейбница о форме кривой, по которой тяжелая точка опускается за равные промежутки времени на равные вертикальные отрезки. Лейбниц и Гюйгенс уже установили, что это полукубическая парабола, но лишь Якоб Бернулли опубликовал доказательство средствами нового анализа, выведя и проинтегрировав дифференциальное уравнение. При этом впервые появился в печати термин <<интеграл>>.

Якоб Бернулли внёс огромный вклад в развитие аналитической геометрии и зарождение вариационного исчисления. Его именем названа лемниската Бернулли. Он исследовал также циклоиду, цепную линию и логарифмическую спираль.

Якобу Бернулли принадлежат значительные достижения в теории рядов, дифференциальном исчислении, теории вероятностей и теории чисел, где его именем названы <<числа Бернулли>>.

Он изучил теорию вероятностей по книге Гюйгенса <<О расчётах в азартной игре>>, в которой ещё не было определения и понятия вероятности. 
Якоб Бернулли ввёл значительную часть современных понятий теории вероятностей и сформулировал первый вариант закона больших чисел. 
Он подготовил монографию в этой области, однако издана она была посмертно в 1713 году его братом Николаем, под названием <<Искусство предположений>> (Ars conjectandi). 
Это содержательный трактат по теории вероятностей, статистике и их практическому применению, итог комбинаторики и теории вероятностей XVII века.
Имя Якоба носит важное в комбинаторике распределение Бернулли.

Якоб Бернулли издал также работы по различным вопросам арифметики, алгебры, геометрии и физики. 

В 1692 году у Якоба Бернулли обнаружились первые признаки туберкулёза, от которого он и скончался 16 августа 1705 года. Похоронен в Базельском соборе. Согласно его пожеланию, на надгробии была изображена логарифмическая спираль.

\newpage
\section{История вопроса}
Первые задачи вероятностного характера возникли в различных азартных играх, таких как кости или карты.
Французский каноник XIII века Ришар де Фурниваль правильно подсчитал все возможные суммы очков после броска трёх костей и указал число способов, которыми может получиться каждая из этих сумм. 
Это число способов можно рассматривать как первую числовую меру ожидаемости события, аналогичную вероятности.
До Фурниваля, а иногда и после него, эту меру часто подсчитывали неверно.

В обширной математической энциклопедии <<Сумма арифметики, геометрии, отношений и пропорций>> итальянца Луки Пачоли (1494) содержатся оригинальные задачи на тему ставок и игр.

Крупный алгебраист XVI века Джероламо Кардано посвятил анализу игры содержательную монографию <<Книга об игре в кости>> (1526 год).
Кардано провёл полный и безошибочный комбинаторный анализ для значений суммы очков и указал для разных событий ожидаемое значение доли <<благоприятных>> событий. Он также сделал проницательное замечание: реальное количество исследуемых событий может при небольшом числе игр сильно отличаться от теоретического, но чем больше игр в серии, тем доля этого различия меньше. 
По существу, Кардано одним из первых близко подошёл к понятию вероятности.

Позднее вместе с Кардано решал задачи Пачоли другой итальянский алгебраист, Никколо Тарталья.

Исследованием данной темы также занимался и Галилео Галилей, написавший трактат <<О выходе очков при игре в кости>> (издана посмертно 1718 год).
Изложение теории игры у Галилея отличается исчерпывающей полнотой и ясностью. 
Кроме того, Галилей привел первые рассуждения, которые стали предсказанием нормального распределения ошибок.

В XVII веке начало формироваться отчётливое представление о проблематике теории вероятностей и появились первые комбинаторные методы решения вероятностных задач. 
Основателями математической теории вероятностей стали Блез Паскаль и Пьер Ферма.
В своей переписке они обсудили ряд проблем, связанных с вероятностными расчётами.

Паскаль в своих трудах далеко продвинул применение комбинаторных методов, которые систематизировал в своей книге <<Трактат об арифметическом треугольнике>> (1665). 

Тематика дискуссии Паскаля и Ферма стала известна Христиану Гюйгенсу, который опубликовал собственное исследование <<О расчётах в азартных играх>> (1657): первый трактат по теории вероятностей, где сам же писал, что речь в трактате не только об играх, а ''здесь закладываются основы очень интересной и глубокой теории.''

Главным достижением Гюйгенса стало введение понятия математического ожидания, то есть теоретического среднего значения случайной величины. Гюйгенс также указал классический способ его подсчёта.

В книге Гюйгенса также присутствует <<задача о разорении игрока>>, правда дана она для самостоятельного решения. Ее полное общее решение дал Абрахам де Муавр полвека спустя (1711). 

Гюйгенс проанализировал и задачу о разделе ставки, дав её окончательное решение. Он также впервые применил вероятностные методы к демографической статистике и показал, как рассчитать среднюю продолжительность жизни.

К этому же периоду относятся публикации английских статистиков Джона Граунта (1662) и Уильяма Петти (1676, 1683). Обработав данные более чем за столетие, они показали, что многие демографические характеристики лондонского населения, несмотря на случайные колебания, имеют достаточно устойчивый характер, например, соотношение числа новорождённых мальчиков и девочек редко отклоняется от пропорции 14 к 13, невелики колебания и процента смертности от конкретных случайных причин. 
Эти данные подготовили научную общественность к восприятию новых идей.

Вопросами теории вероятностей и её применения к демографической статистике занялись также Иоганн Худде и Ян де Витт в Нидерландах, которые в 1671 году составили таблицы смертности и использовали их для вычисления размеров пожизненной ренты. 

На книгу Гюйгенса опирались появившиеся в начале XVIII века трактаты Пьера де Монмора <<Опыт исследования азартных игр>> (опубликован в 1708) и Якоба Бернулли <<Искусство предположений>>.

Над своим трактатом Якоб Бернулли работал двадцать лет, уже лет за десять до публикации текст этого труда в виде незаконченной рукописи стал распространяться по Европе, вызывая большой интерес. 
Трактат стал первым систематическим изложением теории вероятностей. 
Ранее математики чаще всего оперировали самим количеством исходов; историки полагают, что замена количества на <<частоту>> (то есть деление на общее количество исходов) была стимулирована статистическими соображениями: частота, в отличие от количества, обычно имеет тенденцию к стабилизации при увеличении числа наблюдений. Определение вероятности <<по Бернулли>> сразу стало общепринятым.
Единственное важное уточнение, что все <<элементарные исходы>> обязаны быть равновероятны, сделал Пьер-Симон Лаплас в 1812 году. 
Если для события невозможно подсчитать классическую вероятность (например, из-за отсутствия возможности выделить равновероятные исходы), то Бернулли предложил использовать статистический подход, то есть оценить вероятность по результатам наблюдений этого события или связанных с ним.

В этой книге автор привёл, в частности, классическое определение вероятности события как отношения числа исходов, связанных с этим событием, к общему числу исходов. 

В частности, он приводит общую формулу Бернулли.
Также Бернулли подробно излагает комбинаторику и на её основе решает несколько задач со случайным выбором.

Огромное значение как для теории вероятностей, так и для науки в целом имел доказанный Бернулли первый вариант закона больших чисел (название закону дал позже Пуассон). 
Этот закон объясняет, почему статистическая частота при увеличении числа наблюдений сближается с теоретическим её значением -- вероятностью, и тем самым связывает два разных определения вероятности. 
В дальнейшем закон больших чисел трудами многих математиков был значительно обобщён и уточнён; как оказалось, стремление статистической частоты к теоретической отличается от стремления к пределу в анализе -- частота может значительно отклоняться от ожидаемого предела, и можно только утверждать, что вероятность таких отклонений с ростом числа испытаний стремится к нулю. Вместе с тем отклонения частоты от вероятности также поддаются вероятностному анализу.

\newpage
\section{Описание книги}
\subsection{Предисловие}
Автором предисловия к русскому изданию является А.Н.Колмогоров.

В предисловии к книге говорится о том, что именно с закона больших чисел Я.Бернулли и начинается история настоящей теории вероятностей.

Основной его труд <<Искусство предположений>> (или же <<Ars Cnjectandi>>) был издан в Базеле в 1713 году.
Именно в нем была доказана <<теорема Бернулли>> -- первая точно доказанная частная формулировка \textit{закона больших чисел}.

Как уже было сказано, что 
\begin{quote}
	частота появления какого-либо случайного события при неограниченно увеличивающемся числе независимых наблюдений, производимых в одинаковых условиях, сближается с его вероятностью.
\end{quote}
Также замечено, что любая нетривиальная оценка близости между частотой и вероятностью действует не с полной достоверностью, а с вероятностью, меньшей единицы.
Бернулли доказал неравенство, дающую оценку вероятности события
$$\left| \frac{\mu}{N} - p \right| > \varepsilon > 0.$$

Позднее текст <<Искусства предположений>> был переведен на другие языки мира.

В этой же книге <<О законе больших чисел>>, по мимо труда Бернулли публикуется текст юбилейной речи А.А.Маркова.

В 1913 году к юбилею закона больших чисел Академия наук постановила издать в русском переводе четвертую главу, важнейшую часть, труда Бернулли.

Именно о ней дальше и пойдет речь.

\newpage
\subsection{Глава I}
В первой главе, как следует из названия, вводятся предварительные понятия о достоверности, вероятности, необходимости и случайности вещей.

\textit{Достоверность} определяется как 
\begin{quote}
ничто иное, как действительное ее существование в настоящем и будущем; или субъективно, в зависимости от нас, и заключается в степени нашего знания об этом существовании.
\end{quote}
Если про ''объективные'' события, такие как события прошлого, все ясно, то достоверность ''субъективных'' событий, то есть рассматриваемая по отношению к нам, может меняться.
Именно из-за этой менее совершенной оценки событий мы можем говорить о более или менее вероятных событиях.

Далее вводится понятие \textit{вероятности}.
\begin{quote}
\textit{Вероятность} же есть степень достоверности и отличается от нея, как часть целого.
\end{quote}
Полная достоверность будет обозначаться символом $\alpha$ или $1$. 
Приводится пример: так если полная достоверность события будет состоять из пяти вероятностей (как частей), три из которых благоприятствуют исходу, а две другие нет, то вероятность события будет $\frac{3}{5} \alpha$ или же просто $\frac{3}{5}$.

\textit{Более вероятное} событие определяется как событие с большей степенью достоверности.
\textit{Вероятным} событием называется событие, вероятность которого заметно превосходит половину достоверности.
\textit{Возможное} событие -- то, что имеет какую-нибудь степень достоверности (отличную от нуля или бесконечно малой).
\textit{Нравственно достоверно} то событие, вероятность чего почти равна полной достоверности. 
Аналогично определяется и \textit{нравственно невозможное} событие.
\textit{Необходимое} -- 
\begin{quote}
то, что не может не быть в настоящем, будущем или прошедшем и именно по необходимости или \textit{физической} <...>, или \textit{гипотетической} <...>, или по необходимости \textit{условия} или \textit{соглашения} <...>.
\end{quote}
\textit{Случайное}, иначе \textit{свободное} определяется как то, что может быть или не быть в настоящем, прошлом и будущем.
Здесь приводится следующий пример. Если из руки выпустить кость, очевидно, что она упадет, -- несмотря на очевидность, это явление не является необходимым. 
К необходимым событиям Бернулли относит такие явления, как затмения, а явления, как падение кости -- к случайным, так как предполагаемые действия (от которых кость упадет) природе не достаточно известны.
Вводятся еще два интересных определения:
\begin{quote}
\textit{Счастьем} или \textit{несчастьем} называется <...> то, что с большей или, по крайней мере, с равной вероятностью могло бы таковых не принести.
\end{quote}
Так на примере тот, кто нашел клад -- счастлив. 
А несчастлив, к примеру, один из двадцати дезертиров, вытянувший черный жребий на казнь.

\subsection{Глава II}
Данная глава посвящена \textit{предположениям}.

Изначально говорится, что есть вещи, которые твердо известны, то есть мы их \textit{знаем} или \textit{понимаем}. 
В других же случаях мы их \textit{предполагаем}.
Когда мы предполагаем какую-то вещь или событие, то по сути мы измеряем их вероятность. 

Вероятность оценивается доводами, и под \textit{весом} понимается сила этих доводов.
Сами же доводы бывают \textit{внутренними} и \textit{внешними}.
Под внутренними доводами понимаются ''искусственные'', извлекаемые из каких-то соображений, причин, связей, признаков и т.д. 
Внешние доводы, ''неискусственные'', извлекаются из авторитета и свидетельств людей.
Приведен пример:
\begin{quote}
Тит найден убитым на улице.
Мевий обвиняется в совершении убийства.
Доводы обвинения следующие:
	\begin{enumerate}
	\item Известно, что он питал ненависть к Титу (довод от \textit{причины} <...>)
	\item При допросе побледнел и отвечал робко (довод по \textit{действию} <...>)
	\item В доме Мевия найден меч (\textit{признак})
	\item В тот же день, как по дороге был убит Тит, там проходил Мевий  (\textit{обстоятельство} места и времени)
	\item <...> Кай утверждает, что накануне убийства Тита у него была ссора с Мевием (\textit{свидетельство})
	\end{enumerate}
\end{quote}

Прежде чем научиться применять подобные доводы Бернулли приводит общие правила, или же так называемые аксиомы:
	\textit{\begin{quote} 1. Догадкам не место в тех вещах, где можно достигнуть полной достоверности. \end{quote}}
	Так, если судья может проверить показания, он не должен верить на слово подсудимому.
	\textit{ \begin{quote} 2. Не достаточно взвешивать один и другой довод; но нужно добыть все, которые могут дойти до нашего сведения и которые покажутся годными в каком-либо отношении для доказательства предположения. \end{quote}}
	Например, если из порта вышли три корабля, а потом сообщили, что погиб один, не обязательно они могли потонуть с равной вероятностью.
	Стоит иметь в виду другие обстоятельства.
	Так зная, что один был особо старый и поломанный, то он погиб с большей вероятностью	
	\textit{\begin{quote} 3. Следует не только рассматривать доводы, приводящие к утверждению, но и все те, которые могут привести к противоположному заключению, дабы после должного обсуждения тех и других стало ясно, которые перевешивают.\end{quote}}
	К примеру, если о человеке, который отправился в путешествие, на протяжении двадцати лет ничего не известно, нельзя судить, что он мертв. 
	Наравне с доводами о том, что путешественники часто гибнут, или он мог умереть от болезни, нужно рассматривать и противоположные доводы: письмо от него пропало или он просто не хотел писать.
	\textit{\begin{quote} 4. Для суждения о вещах общих достаточны доводы отдаленные и общие; но для суждения о частных вещах следует присоединять также доводы более близкие и специальные, если только такие имеются.\end{quote}}
	Так, если спрашивается насколько вероятно, что 20-ти летний человек переживет 60-летнего, то это будут общие суждения, в расчет которых, кроме этого факта брать нечего.
	Однако, если речь будет идти про конкретных людей, то нужно еще учитывать их состояние здоровья и прочее.
	\textit{\begin{quote} 5. В обстоятельствах неясных и сомнительных действия должны приостанавливаться, пока не прольется больший свет; но если необходимость действия не терпит отлагательства, из двух исходов нужно всегда избирать тот, который кажется более подходящим <...>, хотя бы ни один таковым на деле не был.\end{quote}}
	Здесь приводится пример следующий. Что если при пожаре остается только один способ спастись -- броситься в окно, то стоит выбирать последнее, так как шансов выжить больше.
	Хотя очевидно, что этот способ тоже не дает гарантии сохранения жизни.
	\textit{\begin{quote} 6. Что в некотором случае полезно, но ни в каком не вредно, следует предпочитать тому, что никогда не приносит ни пользы, ни вреда.\end{quote}}
	Эта аксиома, как замечает Бернулли, следствие предыдущей.
	\textit{\begin{quote} 7. Не следует оценивать поступки людей по их результатам. \end{quote}}
	Так например, если некто собирался выбросить тремя бросками на кубиках три шестерки, то он безрассудный, даже если ему это удалось.
	 \textit{\begin{quote} 8. В суждениях наших следует остерегаться, чтобы не приписывать вещам более, чем следует, и не считать самим, а равно и не навязывать другим, за безусловно достоверное, что только вероятнее другого.\end{quote}}
	Здесь поясняется: необходимо, чтобы придаваемая вера соотносилась со степенью достоверности.
	 \textit{\begin{quote} 9. Однако, так как только в редких случаях можно достичь полной достоверности, то необходимость и обычай требуют, чтобы нравственно лишь достоверное событие считалось безусловно достоверным.\end{quote}}
	Здесь Бернулли рассуждает о том, как было бы неплохо, если бы Правительство установило такие ''пределы'' для нравственной достоверности.

\subsection{Глава III}
В этой главе говорится о том, как оцениваются уже упомянутые \textit{веса} доводов для вычисления вероятностей вещей.

Сначала упоминается, что доводы бывают трех видов:
\begin{quote}
	некоторые из них \textit{обязательно существуют, но не обязательно доказывают};
	другие \textit{обязательно существуют, но обязательно доказывают};
	<...> третьи \textit{одновременно не обязательно существуют и не обязательно доказывают}.
\end{quote}
Здесь Бернулли приводит пример: если брат не пишет письмо, то могут быть доводы такие: \textit{леность, смерть, занятия}, которые относятся соответственно к вышеупомянутым видам.

Кроме этого, вводится еще и другая классификация доводов: \textit{чистые} и \textit{смешанные}.
\begin{quote}
	Я называю \textit{чистыми} такие, которые в некоторых случаях так доказывают вещь, что в других ничего не доказывают положительно; 
	\textit{смешанными} же -- такие, которые в известных случаях так доказывают вещь, что в других доказывают противное.
\end{quote}
Здесь также приводятся примеры. 
Так смешанным доводом будет следующее. Если известно, что убийца был в черном, и есть четыре человека в черном (среди них Гракх), то довод, доказывающий, что Гракх -- убийца, смешанный, так как он в том числе говорит о том, что убийцей мог быть и один из трех других.
Довод про то, что тот же Гракх побледнел на допросе, а значит, он убийца,  -- чистый, так как этот довод никак не доказывает его невиновность.

Далее еще раз заостряется внимание на том, что доводы бывают разной силы.
Здесь говорится о том, что вероятность, обусловленную доводом, можно вывести так же, как и исход азартных игр.
\begin{quote}
примем число случаев, при которых некоторый довод может существовать, равным $b$; число случаев, при которых может не существовать, равным $c$; число тех и других $a=b+c$.
Примем также число случаев, при которых довод может доказывать, равным $\beta$ при которых не доказывает или доказывает противное -- равным $\gamma$; число тех и других $\alpha = \beta + \gamma$. 
<...> все случаи одинаково возможны <...>. Иначе необходимо уравнять их и вместо каждого легче встречающегося случая считать столько других, насколько он легче имеет место <...>
\end{quote}
Далее приводятся приводятся простейшие ''формулы'' вычисления весов доводов.
\begin{enumerate}
\item Если довод необязательно существует, но обязательно доказывает, то есть $b$ случаев, которые доказывают, и $c$ случаев, которые не доказывают. Отсюда получается, что довод доказывает следующую часть достоверности: $$\cfrac{b \cdot 1 + c \cdot 0}{a} = \cfrac{b}{a} \, .$$

\item Если довод необходимо существует, но необязательно доказывает, то по аналогичным соображениям получим силу доказательства довода $$\cfrac{\beta \cdot 1 + \gamma \cdot 0}{\alpha} = \cfrac{\beta}{\alpha} \, .$$
Если довод смешанный, то таким же образом получаем величину достоверности противного: $$\cfrac{\gamma \cdot 1 + \beta \cdot 0}{\alpha} = \cfrac{\gamma}{\alpha} \, .$$

\item Если довод необязательно существует и необязательно доказывает, то он доказывает $\frac{\beta}{\alpha}$ вещи, но так как всего случаев $b$. при которых он может существовать, получаем его силу доказательности: $$\cfrac{b \cdot \frac{\beta}{\alpha} + c \cdot 0}{\alpha} = \cfrac{b \beta}{a \alpha}  \, ,$$
и если он смешанный, то сила доказательности противного:$$\cfrac{b \cdot \frac{\gamma}{\alpha} + c \cdot 0}{\alpha} = \cfrac{b \gamma}{a \alpha} \, ,$$

\item Если есть несколько доводов и обозначим:

\hspace{5pt}
\begin{tabular}{l|cccccc}
числа случаев для довода: & 1-го & 2-го & 3-го & $\cdots$  \\
\hline
всех & $a$ & $d$ & $g$ & $p$ & $s$ & $\cdots$ \\
доказывающих & $b$ & $e$ & $h$ & $q$ & $t$ &$\cdots$ \\
не доказывающих & $c$ & $f$ & $i$ & $r$ & $u$ & $\cdots$ \\
\end{tabular}

\hspace{5pt}
Пусть первый довод доказывает $\frac{b}{a}$, второй довод в $e$ случаях доказывает вещь, а в $f$ случаях, не доказывает и оставляет в силе только первый. Тогда в совокупности они доказывают:
$$\cfrac{e \cdot 1 + f \cdot \frac{b}{a}}{d} = \cfrac{(d - f) \cdot 1 +f \cdot \frac{a - c}{a}}{d} = 1 - \frac{cf}{ad} \, .$$

Если присоединяется третий довод, то зная силу доказательности первых двух в совокупности получаем:
$$\cfrac{h \cdot 1 + i \cdot (1 - \frac{cf}{ad})}{g} = \cfrac{(g - i) \cdot 1 + i \cdot \frac{ad - cf}{ad}}{g} = 1 - \frac{cfi}{adg} \, .$$

Аналогично можем получить вероятность и для большего числа доводов.

\item Если все доводы смешанные, то вероятность вещи к вероятности противного относится как $b$ к $c$ у первого, $e$ к $f$ у второго и т.д. 
Тогда отсюда полная доказательная сила из всех трех доводов, т.ч. вероятность  вещи к противоположному равна $beh$ к $cfi$, откуда вероятность вещи и вероятность противоположного соответственно равны:
$$\cfrac{beh}{beh + cfi} \qquad \mbox{и} \qquad \cfrac{cfi}{beh + cfi}\, .$$.

\item Если у нас, часть доводов (пусть три) чистые, а другие (пусть два) смешанные, то достоверность будет вычисляться так. 
По указанному выше первые три довода доказывают в $adj-cfi$ случаях из $adg$.
В остальных $cfi$ случаях смешанные два доказывают с силой достоверности $\frac{qt}{qt+ru}$.
Из выведенного выше все вместе дает вероятность вещи:
$$\cfrac{(adj - cfi) \cdot 1 + cfi \cdot \frac{qt}{qt + ru}}{adg} = 1 - \cfrac{cfi}{adg} \cdot \cfrac{ru}{qt+ru}\, .$$
Замечаем также, что эта вероятность отличается от единицы на произведение отклонения вероятности от единицы для первых доводов ($\frac{cfi}{adj}$) и абсолютной вероятности противоположного из вторых доводов ($\frac{qt}{qt+ru}$).

\item Если кроме доводов <<за>>, есть доводы <<против>>, то доводы обоих видов должны рассматриваться по отдельности по предыдущим правилам.
Может оказаться так, что сумма достоверностей у них может оказаться больше единицы, если доводы сильны, но если нечто имеет $\frac{2}{3}$ достоверности, а противоположное -- $\frac{3}{4}$, то каждая из противоположностей будет вероятной, однако, их вероятности относятся как $8$ к $9$.
\end{enumerate}

В конце главы Бернулли предостерегает от самой распространенной ошибки при использовании правил выше.
Иногда два довода, кажущихся различными, могут представлять собой один и тот же довод.
И наоборот, различные доводы можно принять за один.


\subsection{Глава IV}
В этой главе Бернулли говорит о правильном подсчете числа случаев, а также он формулируют оду важную задачу, которая в последствии даст начало закону больших чисел.

Как вычислять вероятности Бернулли объяснил в предыдущей главе.
Сейчас же, все сводится к тому, чтобы правильно высчитать числа разных случаев и то, насколько одни могут случиться легче других.
Однако, такое сделать удается крайне редко.
Так например, если бы у той же игральной кости были бы грани разных форм и размеров, то вычислить количество событий было бы невозможно.  
Кроме того, нельзя посчитать возможности и невозможность событий будущего.
Также Бернулли приводит много других примеров, когда вычислить достоверно величины невозможно.

Но он делает важное замечание:
\begin{quote}
что не дано вывести a priori, то, по крайней мере, можно получить a posteriori, т.е. из многократного наблюдения результатов в подобных примерах.
\end{quote}
Так например, наблюдая за погодой, можно рассчитать отношения ясных дней к пасмурным для данной местности.

Бернулли заостряет внимание на том, что для таких выводов не достаточно взять несколько наблюдений -- нужно сделать большой запас из них. 
Несмотря на то, что это замечание всем интуитивно понятно, ученый задумывается о том, как при увеличении числа наблюдений меняется действительное отношение между числами случаев: может ли вероятность достижения истинного отношения превзойти степень достоверности или у нее есть некий ''предел''.
Для ясности: требуется узнать, можно ли (и сколько раз) вытаскивать по камешку из урны, в которой три тысячи белых и две тысячи черных камешков, чтобы было нравственно достоверным, что вероятности появления черного камешка и белого относятся как 3 к 2.
В следующей главе это будет доказано.
И взяв это во внимание, то можно считать, что найденные случаи a posteriori с той же точностью, что и известные вероятности a priori.

Бернулли придает очень важное значение этой задаче с шарами, так как за счет всевозможных наблюдений в больших количествах (например, за распространением болезней, погодой и пр.) можно делать очень важные выводы с какой-то небольшой погрешностью (Бернулли ее называет <<известной степенью приближенного>>).
Так например, вытаскивая шары, можно получить отношения $\frac{301}{200}$ и $\frac{299}{200}$ или $\frac{3001}{200}$ и $\frac{2999}{200}$.
Тут говорится о том, что можно сделать так, что отношение будет уклоняться от $\frac{3}{2}$ не больше, чем на заданную величину.

В последних абзацах главы Бернулли ''защищается от возражений'' по поводу того, что задачи про камешки и число болезней похожи.
Так, по его мнению отношение камешков и отношения болезней и природных явлений в равной степени могут считаться неопределенными и неясными, хотя может показаться, что первое число весьма определенное.

Он также отвечает и на то, что многие могут заявить, что число камешков конечное, а болезней и пр. -- бесконечно.
Здесь он напоминает, что даже между двумя бесконечностями можно вывести отношение.

Кроме того, в отличии от камешков, число болезней может увеличиваться.
На это Бернулли отвечает, что иногда лишь нужно возобновлять наблюдения, и из устаревших данных не стоит делать современных выводов.


\subsection{Глава V}
В данной главе Бернулли излагает решение той самой задачи.

Сначала он хочет привести несколько лемм, а затем все, что останется, -- их применить.
\begin{lemma} \label{lem1}
Пусть дан бесконечный ряд последовательных чисел, начиная от нуля, крайнее и наибольшее из которых $r+s$, среднее $r$ и соседние к среднему $r - 1$ и $r+1$. 
Затем ряд продолжается до кратного к $r+s$, то есть до $nr+ns$, а среднее и соседние к нему увеличиваются в том же отношении:
$$0, 1, 2, 3, \dots, nr-n, \dots, nr, \dots, nr+n, \dots, nr+ns.$$
Утверждается, что число членов последовательности между $nr+n$ и $nr+ns$ не будет более,чем в $s-1$ раз, и число членов последовательности между $0$ и $nr-n$ не будет более, чем в $r-1$ раз, превышать число членов между $nr$ и $nr+n$ (или между $nr-n$ и $nr$).
\end{lemma}
Доказательство сводится к тому, что разница членов между ними -- $ns-n$, $nr-n$ и $n$ соответственно, а их отношение -- $s-1 : r-1 : 1$.

\begin{lemma} \label{lem2}
Всякая степень $r+s$ выражается числом членов на один больше, чем степень.
\end{lemma}
Это следует из того, что квадрат числа содержит 3 члена, куб -- 4, и т.д.
 
\begin{lemma} \label{lem3}
В любой степени двучлена $r+s$ (по крайне мере в такой, что показатель вида $nr+ns$, где $n$ -- целое) некоторый член $M$ -- наибольший, если число предшествующих и последующих членов находятся в отношении $s$ к $r$, или что равносильно, показатели $r$ и $s$ относятся как сами $r$ к $s$.
\end{lemma}
\begin{proof}[Доказательство.]
 Известно, что $$(r+s)^{nt} = r^{nt}+\frac{nt}{1} r^{nt-1}s + \frac{nt (nt-1)}{1 \cdot 2}r^{nt-2}s^2 + \frac{nt (nt-1) (nt-2)}{1 \cdot 2 \cdot 3} r^{nt-3} s^3 + \,\dots.$$

Видно, что степени $r$ уменьшатся, а степени $s$ увеличиваются, причем коэффициенты у $k$-ого члена с начала и с конца совпадают.
Так как по Лемме \ref{lem2} число членов без $M$ равно $nt=nr+ns$, а предположению число предшествующих и последующих за ним относится как $s : r$, то непосредственно их количество равно $ns$ и $nr$.
Значит, $$M = \frac{nt (nt-1) \dots (nr+1)}{1 \cdot 2 \dots ns} r^{nr} s^{ns} = 
\frac{nt (nt-1) \dots (ns+1)}{1 \cdot 2 \dots nr} r^{nr} s^{ns} $$

Отсюда следует, что члены слева можно записать как
$$\frac{nt (nt-1) \dots (nr+2)}{1 \cdot 2 \dots (ns-1)} r^{nr+1} s^{ns-1}, \quad \frac{nt (nt-1) \dots (nr+3)}{1 \cdot 2 \dots (ns-2)} r^{nr+2} s^{ns-2}, \quad \dots ;$$
а члены справа:
$$\frac{nt (nt-1) \dots (ns+2)}{1 \cdot 2 \dots (nr-1)} r^{nr-1} s^{ns+1}, \quad \frac{nt (nt-1) \dots (ns+3)}{1 \cdot 2 \dots (nr-2)} r^{nr-2} s^{ns+2}, \quad \dots $$

Далее, все, что нужно, это посмотреть на отношения выписанных членов.
Во-первых, видно, что более ближний член к $M$ больше более дальнего с той же стороны.
Так же видно, что $M$ больше своих соседних, откуда следует, что $M$ -- наибольший.

Во-вторых, отношение между $M$ и близким к нему членам меньше, чем отношение между более близким к более удаленному (при равном числе промежуточных членов).

Что и требовалось доказать.
\end{proof}

\begin{lemma} \label{lem4}
В степени $r+s$ с показателем $nt$ число $n$ может быть взято столь большим, чтобы отношение наибольшего члена $M$ к любому из двух других $L$ и $\Lambda$, отстоящих от него на $n$ членов с двух сторон, превзошло всякое данное отношение.
\end{lemma}
\begin{proof}[Доказательство.]
Из Леммы \ref{lem3} $$M = \frac{nt (nt-1) \dots (nr+1)}{1 \cdot 2 \dots ns} r^{nr} s^{ns} = 
\frac{nt (nt-1) \dots (ns+1)}{1 \cdot 2 \dots nr} r^{nr} s^{ns}.$$

Тогда, $$L = \frac{nt (nt-1) \dots (nr+n+1)}{1 \cdot 2 \dots (ns-n)} r^{nr+n} s^{ns-n},$$
$$\Lambda = \frac{nt (nt-1) \dots (ns+n+1)}{1 \cdot 2 \dots (nr-n)} r^{nr-n} s^{ns+n}.$$

Тогда отношения равны следующим выражениям:
$$\frac{M}{L} = \frac{(nr+n) (nr+n -1) \dots (nr+1) \cdot s^n}{(ns-n + 1) (ns-n+2) \dots ns \cdot r^n}= \frac{(nrs+ns) (nrs+ns-s) \dots (nrs+s)}{(nrs-nr+r) (nrs-nr+2r) \dots nrs},$$
$$\frac{M}{\Lambda} = \frac{(ns+n) (ns+n -1) \dots (ns+1) \cdot r^n}{(nr-n + 1) (nr-n+2) \dots nr \cdot s^n}= \frac{(nrs+nr) (nrs+nr-s) \dots (nrs+r)}{(nrs-ns+s) (nrs-ns+2s) \dots nrs}.$$

При бесконечном $n$ эти отношения также бесконечно большие, так как составлены из бесконечного произведения. 
Так же Бернулли здесь упоминает \textit{Трактат о бесконечных рядах} и советует читателю обратиться к нему.
\end{proof}

\begin{lemma} \label{lem5}
В условиях Леммы \ref{lem4} можно привести такое большое $n$, чтобы сумма всех членов от среднего и наибольшего $M$ до любого из членов $L$ и $\Lambda$ включительно имела к сумме всех других вне пределов $L$ и $\Lambda$ отношение, большее всякого заданного числа.
\end{lemma}
\begin{proof}[Доказательство.]
Пусть члены между $L$ и $M$ обозначаются как $F, G, H, \dots$, а за пределом $L$: $P, Q, R, \dots$.
Из Леммы \ref{lem3} следует:
$$\frac{M}{F} < \frac{L}{P}, \quad \frac{F}{G} < \frac{P}{Q}, \quad \frac{G}{H} < \frac{Q}{R}, \quad \dots$$

Из Леммы \ref{lem4} (рассуждение про бесконечность отношения при бесконечном $n$) следует,что бесконечно отношение
$$\frac{F+G+H+\dots}{P+Q+R+\dots},$$
иначе говоря, сумма между $M$ и $L$ бесконечно больше суммы за $L$.
Из Леммы \ref{lem1} число членов за $L$ превышает не более чем в $s-1$ раз, а сами члены уменьшаются, чем дальше они отстоят от $L$.

Аналогичные рассуждения для $\Lambda$.

И уже напрямую отсюда следует утверждение леммы.
\end{proof}

Далее Бернулли во избежании рассуждений с <<бесконечностями>> рассказывает, как найти такое $n$, чтобы отношения из Леммы \ref{lem5} были больше некого $c$.

Для этого он берет такое $m$, чтобы стало верным 
$$\left(\frac{r+1}{r}\right)^m \geqslant c (s-1).$$
Отсюда находится $$m \geqslant \frac{\log (c(s-1))}{\log (r+1) - \log r}.$$

Далее, Бернулли ищет такое $n$, чтобы дробь порядка $m$ (взятая из произведения в лемме) стала равной $\frac{r+1}{r}$:
$$\frac{nrs+ns-ms+s}{nrs-nr+mr} = \frac{r+1}{r}$$ 
$$n = m + \frac{ms-s}{r+1}.$$
Затем говорится, что при показателе $nt = mt+\frac{mst-st}{r+1}$ у $r+s$, наибольший член будет более, чем в $c(s-1)$ раз превосходить  $L$.
Это следует из того, все предыдущие дроби больше $\frac{r+1}{r}$, а значит, умножив эту дробь саму на себя $m$ раз, получим число, большее $c(s-1)$.

Далее, так как $$\frac{M}{L} < \frac{F}{P} < \frac{G}{Q} < \dots,$$
то каждый член за $M$ превзойдет соответствующий член за $L$ более, чем в $c(s-1)$ раз, а как следствие, сама больше хотя бы во столько раз и соответствующая сумма.

Так как число членов за $L$ превосходит число членов за $M$ не более, чем в $s-1$ раз, то и искомое отношение хотя бы $c$, что и требовалось доказать.

Для членов справа Бернулли приводит аналогичное доказательство.

Далее будет приведено предложение, которое Бернулли называет ''главным'', но сначала приводятся определения.
Случай, когда какое-либо событие появляется, -- \textit{плодовитый (благоприятный)}.
Случай, когда то же событие не появляется, -- \textit{бесплодный (неблагоприятный)}.
Опыт -- \textit{благоприятный (неблагоприятный)}, когда обнаруживается хотя бы один из благоприятных (неблагоприятных) случаев.

\begin{proposition}
Пусть число благоприятных случаев относится к числу неблагоприятных как $r$ к $s$, или к числу всех случаев как $r$  к $t=r+s$, а это отношение заключается в пределах $\cfrac{r+1}{t}$ и $\cfrac{r-1}{t}$.
Утверждается, что можно взять столько опытов, чтобы в какое угодно данное число раз $c$ было вероятнее, что число благоприятных опытов попадет в эти пределы, а не вне их, т.е. отношение числа благоприятных наблюдений к числу всех не более $\cfrac{r+1}{t}$ и не менее $\cfrac{r-1}{t}$.
\end{proposition}

\begin{proof}[Доказательство.]
Пусть число необходимых наблюдений $nt$.

Здесь Бернулли говорит слова, очень похожие на современное понимание закона больших чисел: что большое число случайных экспериментов в совокупности дадут результат, который не зависит от случая.

Требуется найти вероятность того, что неблагоприятных наблюдений будет 0, 1, 2, 3 и т.д.

Вероятность того, что неблагоприятных событй нет: $r^{nt} : t^{nt}$;\\
вероятность одного неблагоприятного события -- $\cfrac{nt}{1} r^{nt-1} s : t^{nt}$;\\
двух -- $\cfrac{nt(nt-1)}{1 \cdot 2} r^{nt-2} s^2 : t^{nt}$;\\
трех -- $\cfrac{nt(nt-1)(nt-2)}{1 \cdot 2 \cdot 3} r^{nt-3} s^3 : t^{nt}$, и т.д.

Заметим, что если отбросить делитель $t^{nt}$, получится ровно тот же ряд, который рассматривался в предыдущих леммах.
Из Леммы \ref{lem3} следует, что наибольший член соответствует $nr$ благоприятным и $ns$ неблагоприятным опытам.
В соответствие $L$ и $\Lambda$ можно поставить числа случаев с $nr+n$ и $nr-n$ благоприятными наблюдениями.
Суммарно число случаев между ними выражается суммой между $L$ и $\Lambda$.
Из Лемм \ref{lem4} и \ref{lem5} следует, что можно взять такое число наблюдений, что число случаев ,когда отношение благоприятных ко всем лежит в границах из условия, превышало более чем в  $c$ раз число остальных случаев.
Что и требовалось доказать.
\end{proof}

Далее следует рассуждение, что чем больше взять количество опытов, тем уже можно сделать границы $\cfrac{r+1}{t}$ и $\cfrac{r-1}{t}$.

Напоследок Бернулли приводит пример. 
Так если из опытов должно выйти, что $\frac{r}{s} = \frac{3}{2}$, то стоит брать $r$ и $s$ равные не 3 и 2, а 30 и 20, 300 и 200 и т.д.
Так он высчитывает, что если взять $c=1000$,  то при $25550$ тысяч опытах более, чем в тысячу раз вероятнее, что  отношение числа благоприятных наблюдений к числу всех лежит между $\frac{31}{50}$ и $\frac{29}{50}$.
Затем Бернулли также приводит еще подобные числа для $c=10000$ и $c=100000$, где видно, что при увеличении числа опытов увеличивается вероятность того же самого в 10000 раз, в 100000 раз.

И наконец, он приходит к тому, что если продолжать опыты до бесконечности, то получили бы точное отношение и вероятность совпала бы с достоверностью.


\newpage
\section{Выводы}
Несмотря на то, что трактат <<Искусство предположений>> Якоба Бернулли остался незавершенным, этот труд стал первым изложением теории вероятностей. 
Несомненно, Бернулли заложил огромный фундамент для развития теории вероятностей.

Здесь приводится определение самой \textit{вероятности} в том понимании, которое у нас сейчас.
Также стоит отметить, что до Бернулли математики в основном пользовались количеством исходов, а после него все математики стали оперировать вероятностью.

Нельзя не отметить и рассуждения Бернулли, где он переходит от классического определения вероятности к так называемому ''статистическому''.

Кроме того, в третьей главе Бернулли привел формулы вычисления вероятностей (весов доводов). Сейчас нам эти формулы кажутся понятными и не требуют объяснений, однако до Гюйгенса и Бернулли многие делали подобные вычисления неверно.

Огромное значение для всей математики в целом имеет доказанная в пятой главе первая предельная теорема -- закон больших чисел.
Этот закон объясняет, почему при увеличении числа опытов частота стремится к вероятности.

Как мы знаем, в последствии благодаря трудам других математиков (Хинчин, Колмогоров, Чебышев и др.) появилось много обобщений теоремы Бернулли, но именно Бернулли положил основу всей теории вероятности.

Несмотря на то, что труд был опубликован в начале XVIII века, он остается актуальным и для современной математики. 
Мы до сих пор пользуемся определениями, предложенными Бернулли, изучаем биномиальное распределение.
Однако закон Бернулли сейчас формулируется в несколько другом виде:
$$\lim\limits_{\nu \to \inf} P \left\{ \left| \frac{\mu}{\nu} - p\right| < \varepsilon \right\} = 1,$$
 где $p$ -- вероятность события, $\mu$ -- количество появления в $\nu$ испытаниях.
 
Чтение этой книги мне показалось весьма интересным и познавательным. 
Очевидно, формат самого текста отличается от современной математической литературы.
Было интересно замечать, как Бернулли, исходя каких-то ''житейских'' определений, вводит формальные математические понятия.
В первых четырех главах мало сухого математического текста: Бернулли стремится донести до каждого читателя определения, приводя в соответствие каждому определению примеры из жизни.

Однако также, я бы отметила некоторую сложность при чтении: русский текст был написан в начале $XX$ века, а тогда и построение предложений, и окончания существенно отличались от таковых в настоящее время.

Тем не менее, я не пожалела, что выбрала этот труд для написания аннотации, ведь его важность трудно переоценить.
 
\newpage
\begin{thebibliography}{99}
\bibitem{1} \textsc{Бернулли Я.} О законе больших чисел. М., Наука, 1986
\bibitem{2} \textsc{Юшкевич А.П. и др.} История математики. Т.3. Математика XVIII столетия. М., Наука, 1972
\bibitem{3} https://ru.wikipedia.org/wiki/Бернулли,\_Якоб 

\end{thebibliography}

\end{document}