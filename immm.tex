\documentclass[12pt]{article}
\usepackage[utf8]{inputenc}
\usepackage[T2A]{fontenc}
\usepackage[russian]{babel}
\usepackage{hyperref}
\usepackage{cite,enumerate,float,indentfirst}

\usepackage{amsthm}
\usepackage{amsbsy}
\usepackage{amsmath}
\usepackage{amssymb}
\usepackage{amsfonts}
\usepackage{mathtext}
\usepackage{float}

\usepackage{array}
\usepackage{longtable}

\usepackage[pdftex]{graphicx}
\usepackage{graphics}
\usepackage{xcolor}
\usepackage{tikz}
\usepackage{verbatim}

\newtheorem{example}{Пример}[section]
\newtheorem{definition}{Определение}[section]
\newtheorem{proposition}{Предложение}[section]
\newtheorem{theorem}{Теорема}[section]
\newtheorem{corollary}{Следствие}[section]
\newtheorem{lemma}{Лемма}[section]

\DeclareMathOperator{\tr}{tr}

\newcommand{\dif}[2]{\frac{\partial #1}{\partial #2}}

%\parindent = 30pt
%\hoffset = 0pt
%\voffset = 0pt
%\oddsidemargin = 72pt
%\topmargin = 18pt
%\headheight = 12pt
%\headsep = 25pt
%\textheight = 568pt
%\textwidth = 360pt
%\marginparsep = 10pt
%\marginparwidth = 103pt
%\paperwidth = 597pt
%\paperheight = 845pt


\begin{document}

\renewcommand{\contentsname}{Содержание}

\begin{titlepage} \newpage 
	\begin{center} МОСКОВСКИЙ ГОСУДАРСТВЕННЫЙ УНИВЕРСИТЕТ ИМ. М.В.ЛОМОНОСОВА\\ \end{center} 
	\vspace{3em} 
	\begin{center} \Large Механико-математический факультет \\ \end{center}
	\vspace{5em} 
	\begin{center} 
		\textsc{
			\textit{Аннотация к книге} \linebreak
			\textbf{
				Я.Бернулли \linebreak
				\large{О законе больших чисел}
			}
		}
	\end{center}
	\vspace{6em} \newbox{\lbox} \savebox{\lbox}{\hbox{Нагорных Яна}} 
	\newlength{\maxl} \setlength{\maxl}{\wd\lbox} \hfill\parbox{11cm}
	{ \hspace*{5cm}\hspace*{-5cm}Студент:\hfill\hbox to\maxl{Нагорных Я.В.}\\
	\hspace*{5cm}\hspace*{-5cm}Преподаватель:\hfill\hbox to\maxl{Подколзина М.А.}\\ \\
	\hspace*{5cm}\hspace*{-5cm}Группа:\hfill\hbox to\maxl{410}\\ } 
	\vspace{\fill}
	\begin{center} Москва \\ 2018 \end{center} 
\end{titlepage}

\tableofcontents

\newpage
\section{Краткая биография}
Якоб Бернулли, 6 января 1655, Базель, — 16 августа 1705, там же) — швейцарский математик. Один из основателей теории вероятностей и математического анализа. Старший брат Иоганна Бернулли, совместно с ним положил начало вариационному исчислению. Доказал частный случай закона больших чисел — теорему Бернулли[6]. Профессор математики Базельского университета (с 1687 года)[6]. Иностранный член Парижской академии наук (1699) и Берлинской академии наук (1702).


Якоб Бернулли родился 6 января 1655 года в швейцарском Базеле.
Младший брат Якоба Иоганн также стал математиком.
Его отец Николай был преуспевающим фармацевтом. 

По его же желанию Якоб пошел учиться богословию в Базельский университет, однако увлёкся математикой, которую изучил самостоятельно. 
В университете Якоб выучил французский, итальянский, английский, латинский и греческий языки.
В 1671 году он получил учёную степень магистра философии.
Позднее Бернулли много путешествовал по Европе, что дало ему знакомства с Гюйгенсом, Гуком и Бойлем.

В 1683 году вернулся в Базель после путешествий и вскоре женился на Юдит Штупанус. Позднее у них родились сын и дочь.

В 1683 году начал читать лекции по физике в Базельском университете. 
С 1687 года избран профессором математики в этом университете и работал там до конца жизни.
В то же время он начал изучать труды Лейбница по анализу и с энтузиазмом начал освоение нового исчисления. Также он привлек в изучение анализа своего младшего брата Иоганна.
Позднее Лейбниц откликнулся на письмо Якоба, и их взаимная переписка обогатила зарождающийся анализ.

В 1699 году Якоб Бернулли стал членом Парижской академии наук, а в 1702 году -- Берлинской академии наук.

Первое триумфальное выступление молодого математика относится к 1690 году. Якоб решает задачу Лейбница о форме кривой, по которой тяжелая точка опускается за равные промежутки времени на равные вертикальные отрезки. Лейбниц и Гюйгенс уже установили, что это полукубическая парабола, но лишь Якоб Бернулли опубликовал доказательство средствами нового анализа, выведя и проинтегрировав дифференциальное уравнение. При этом впервые появился в печати термин <<интеграл>>.

Якоб Бернулли внёс огромный вклад в развитие аналитической геометрии и зарождение вариационного исчисления. Его именем названа лемниската Бернулли. Он исследовал также циклоиду, цепную линию, и особенно логарифмическую спираль.

Якобу Бернулли принадлежат значительные достижения в теории рядов, дифференциальном исчислении, теории вероятностей и теории чисел, где его именем названы <<числа Бернулли>>.

Он изучил теорию вероятностей по книге Гюйгенса <<О расчётах в азартной игре>>, в которой ещё не было определения и понятия вероятности. 
Якоб Бернулли ввёл значительную часть современных понятий теории вероятностей и сформулировал первый вариант закона больших чисел. 
Он подготовил монографию в этой области, однако издана она была посмертно в 1713 году его братом Николаем, под названием <<Искусство предположений>> (Ars conjectandi). 
Это содержательный трактат по теории вероятностей, статистике и их практическому применению, итог комбинаторики и теории вероятностей XVII века.
Имя Якоба носит важное в комбинаторике распределение Бернулли.

Якоб Бернулли издал также работы по различным вопросам арифметики, алгебры, геометрии и физики. 

В 1692 году у Якоба Бернулли обнаружились первые признаки туберкулёза, от которого он и скончался 16 августа 1705 года. Похоронен в Базельском соборе. Согласно его пожеланию, на надгробии была изображена логарифмическая спираль.

\newpage
\section{История вопроса}
Первые задачи вероятностного характера возникли в различных азартных играх, таких как кости или карты.
Французский каноник XIII века Ришар де Фурниваль правильно подсчитал все возможные суммы очков после броска трёх костей и указал число способов, которыми может получиться каждая из этих сумм. 
Это число способов можно рассматривать как первую числовую меру ожидаемости события, аналогичную вероятности.
До Фурниваля, а иногда и после него, эту меру часто подсчитывали неверно.

В обширной математической энциклопедии <<Сумма арифметики, геометрии, отношений и пропорций>> итальянца Луки Пачоли (1494) содержатся оригинальные задачи на тему ставок и игр.

Крупный алгебраист XVI века Джероламо Кардано посвятил анализу игры содержательную монографию <<Книга об игре в кости>> (1526 год).
Кардано провёл полный и безошибочный комбинаторный анализ для значений суммы очков и указал для разных событий ожидаемое значение доли <<благоприятных>> событий. Он также сделал проницательное замечание: реальное количество исследуемых событий может при небольшом числе игр сильно отличаться от теоретического, но чем больше игр в серии, тем доля этого различия меньше. 
По существу, Кардано одним из первых близко подошёл к понятию вероятности:

Позднее вместе с Кардано решал задачи Пачоли другой итальянский алгебраист, Никколо Тарталья.

Исследованием данной темы также занимался и Галилео Галилей, написавший трактат <<О выходе очков при игре в кости>> (издана посмертно 1718 год).
Изложение теории игры у Галилея отличается исчерпывающей полнотой и ясностью. 
Кроме того, Галилей привел первые рассуждения, которые стали предсказанием нормального распределения ошибок.

В XVII веке начало формироваться отчётливое представление о проблематике теории вероятностей и появились первые комбинаторные методы решения вероятностных задач. 
Основателями математической теории вероятностей стали Блез Паскаль и Пьер Ферма.
В своей переписке они обсудили ряд проблем, связанных с вероятностными расчётами.

Паскаль в своих трудах далеко продвинул применение комбинаторных методов, которые систематизировал в своей книге <<Трактат об арифметическом треугольнике>> (1665). 

Тематика дискуссии Паскаля и Ферма стала известна Христиану Гюйгенсу, который опубликовал собственное исследование <<О расчётах в азартных играх>> (1657): первый трактат по теории вероятностей, где сам же писал, что речь в трактате не только об играх, а ''здесь закладываются основы очень интересной и глубокой теории.''

Главным достижением Гюйгенса стало введение понятия математического ожидания, то есть теоретического среднего значения случайной величины. Гюйгенс также указал классический способ его подсчёта.

В книге Гюйгенса также присутствует <<задача о разорении игрока>>, правда дана она для самостоятельного решения. Ее полное общее решение дал Абрахам де Муавр полвека спустя (1711). 

Гюйгенс проанализировал и задачу о разделе ставки, дав её окончательное решение. Он также впервые применил вероятностные методы к демографической статистике и показал, как рассчитать среднюю продолжительность жизни.

К этому же периоду относятся публикации английских статистиков Джона Граунта (1662) и Уильяма Петти (1676, 1683). Обработав данные более чем за столетие, они показали, что многие демографические характеристики лондонского населения, несмотря на случайные колебания, имеют достаточно устойчивый характер, например, соотношение числа новорождённых мальчиков и девочек редко отклоняется от пропорции 14 к 13, невелики колебания и процента смертности от конкретных случайных причин. 
Эти данные подготовили научную общественность к восприятию новых идей.

Вопросами теории вероятностей и её применения к демографической статистике занялись также Иоганн Худде и Ян де Витт в Нидерландах, которые в 1671 году составили таблицы смертности и использовали их для вычисления размеров пожизненной ренты. 

На книгу Гюйгенса опирались появившиеся в начале XVIII века трактаты Пьера де Монмора <<Опыт исследования азартных игр>> (опубликован в 1708) и Якоба Бернулли <<Искусство предположений>>.

Над своим трактатом Якоб Бернулли работал двадцать лет, уже лет за десять до публикации текст этого труда в виде незаконченной рукописи стал распространяться по Европе, вызывая большой интерес. 
Трактат стал первым систематическим изложением теории вероятностей. 
Ранее математики чаще всего оперировали самим количеством исходов; историки полагают, что замена количества на ''частоту'' (то есть деление на общее количество исходов) была стимулирована статистическими соображениями: частота, в отличие от количества, обычно имеет тенденцию к стабилизации при увеличении числа наблюдений. Определение вероятности ''по Бернулли'' сразу стало общепринятым.
Единственное важное уточнение, что все <<элементарные исходы>> обязаны быть равновероятны, сделал Пьер-Симон Лаплас в 1812 году. 
Если для события невозможно подсчитать классическую вероятность (например, из-за отсутствия возможности выделить равновероятные исходы), то Бернулли предложил использовать статистический подход, то есть оценить вероятность по результатам наблюдений этого события или связанных с ним.

В этой книге автор привёл, в частности, классическое определение вероятности события как отношения числа исходов, связанных с этим событием, к общему числу исходов. 
Систематически изученная Бернулли вероятностная схема сейчас называется биномиальным распределением.

В частности, он приводит общую ''формулу Бернулли''. 
Также Бернулли подробно излагает комбинаторику и на её основе решает несколько задач со случайным выбором.

Огромное значение как для теории вероятностей, так и для науки в целом имел доказанный Бернулли первый вариант закона больших чисел (название закону дал позже Пуассон). 
Этот закон объясняет, почему статистическая частота при увеличении числа наблюдений сближается с теоретическим её значением -- вероятностью, и тем самым связывает два разных определения вероятности. 
В дальнейшем закон больших чисел трудами многих математиков был значительно обобщён и уточнён; как оказалось, стремление статистической частоты к теоретической отличается от стремления к пределу в анализе -- частота может значительно отклоняться от ожидаемого предела, и можно только утверждать, что вероятность таких отклонений с ростом числа испытаний стремится к нулю. Вместе с тем отклонения частоты от вероятности также поддаются вероятностному анализу.

\newpage
\section{Описание книги}
\subsection*{Предисловие}
В предисловии к книге говорится о том, что именно с закона больших чисел Я.Бернулли и начинается история настоящей теории вероятностей.

Основной его труд <<Искусство предположений>> (или же <<Ars Cnjectandi>>) был издан в Базеле в 1713 году.
Именно в нем была доказана ''теорема Бернулли'' -- первая точно доказанная частная формулировка \textit{закона больших чисел}.

Как уже было сказано, что 
\begin{quote}
	частота появления какого-либо случайного события при неограниченно увеличивающемся числе независимых аблюдений, производимых в одинаковых условиях, сближается с его вероятностью.
\end{quote}
Также замечено, что любая нетривиальная оценка близости между частотой и вероятностью действует не с полной достоверностью, а с вероятностью, меньшей единицы.
Бернулли доказал неравенство, дающую оценку вероятности события
$$\left| \frac{\mu}{N} - p \right| > \varepsilon > 0.$$

Позднее текст <<Искусства предположений>> был переведен на другие языки мира.

В этой же книге <<О законе больших чисел>>, по мимо труда Бернулли публикуется текст юбилейной речи А.А.Маркова.

В 1913 году к юбилею закона больших чисел Академия наук постановила издать в русском переводе четвертую главу, важнейшую часть, труда Бернулли.

\newpage
\subsection{Глава I}
В первой главе, как следует из названия, вводятся предварительные понятия о достоверности, вероятности, необходимости и случайности вещей.

\textit{Достоверность} определяется как 
\begin{quote}
ничто иное, как действительное ее существование в настоящем и будущем; или субъективно, в зависимости от нас, и заключается в степени нашего знания об этом существовании.
\end{quote}
Если про ''объективные'' события, такие как события прошлого, все ясно, то достоверность ''субъективных'' событий, то есть рассматриваемая по отношению к нам, может меняться.
Именно из-за этой менее совершенной оценки событий мы можем говорить о более или менее вероятных событиях.

Далее вводится понятие \textit{вероятности}.
\begin{quote}
\textit{Вероятность} же есть степень достоверности и отличеятся о тнея, как часть целого.
\end{quote}
Полная достоверность будет обозначаться символом $\alpha$ или $1$. 
Приводится пример: так если полная достоверность события будет состоять из пяти вероятностей (как частей), три из которых благоприятствуют исходу, а две другие нет, то вероятность события будет $\frac{3}{5} \alpha$ или же просто $\frac{3}{5}$.

\textit{Более вероятное} событие определяется как событие с большей степенью достоверности.
\textit{Вероятным} событием называетя событие, вероятность которого заметно превосходит половину достоверности.
\textit{Возможное} событие -- то, что имеет какую-нибудь степень достоверности (отличную от нуля или бесконечно малой).
\textit{Нравственно достоверно} то событие, вероятность чего пости равна полной достоверности. 
Аналогично определяется и \textit{нравственно невозможное} событие.
\textit{Необходимое} -- 
\begin{quote}
то, что не может не быть в настоящем, будущем или прошедшем и именно по необходимости или \textit{физической} <...>, или \textit{гипотетической} <...>, или по необходимости \textit{условия} или \textit{соглашения} <...>.
\end{quote}
\textit{Случайное}, иначе \textit{свободное} определяется как то, что может быть или не быть в настоящем, прошлом и будущем.
Здесь приводится следующий пример. Несмотря на то, что если из руки выпустить кость, очевидно, что она упадет, это явление не является необходимым. К необходимым событиям Бернулли относит такие явления, как затмения, а явления, как падение кости -- к случайным, так как предполагаемые действия (от которых кость упадет) природе не достаточо известны.
Вводятся еще два интересных определения:
\begin{quote}
\textit{Счастьем} или \textit{несчастьем} называется <...> то, что с большей или, по крайней мере, с равной вероятностью могло бы таковых не принести.
\end{quote}
Так на примере тот, кто нашел клад -- счастлив. 
А несчастлив, к примеру, вытянувший черный жребий на казнь один из двадати дезертиров.

\section{Глава II}
Данная глава посвящена \textit{предположениям}.

Изначально говорится, что есть вещи, который твердо известны, то есть мы их \textit{знаем} или \textit{пониаем}. 
В других же случаях мы их \textit{предполагаем}.
Когда мы предполагаем какую-то вещь или событие, то по сути мы измеряем их вероятность. 

Вероятность оценивается доводами, и под \textit{весом} понимается сила этих доводов.
Сами же доводы бывают \textit{внутренними} и \textit{внешними}.
Под внутренними доводами понимаются ''искусственные'', извлекаемые из каких-то соображений, причин, связей, признаков и т.д. 
Внешние доводы, ''неискусственные'', извлекаются из авторитета и свидетельств людей.
Приведен пример:
\begin{quote}
Тит найден убитым на улице.
Мевий обвиняется в совершении убийства.
Доводы обвинения следующие:
	\begin{enumerate}
	\item Известно, что он питал ненависть к Титу (довод от \textit{причины} <...>)
	\item При допросе побледнел и отвечал робко (довод по \textit{действию} <...>)
	\item В доме Мевия найден меч (\textit{признак})
	\item В тот же день, как по дороге был убит Тит, там проходил Мевий  (\textit{обстоятельство} места и времени)
	\item <...> Кай утверждает, что накануне убийства Тита у него была ссора с Мевием (\textit{свидетельство})
	\end{enumerate}
\end{quote}

Прежде чем научиться применять подобные доводы Бернулли приводит общие правила, или же так называемые аксиомы:
\begin{enumerate}
	\item \textit{Догадкам не место в тех вещах, где можно достигнуть полной достоверности.}
	Так, если судья может проверить показания, он не должен верить на слово подсудимому.
	\item \textit{Не достаточно взвешивать один и другой довод; но нужно добыть все, которые могут дойти до нашего сведения и которые покажутся годными в каком-либо отношении для доказательства предположения.}
	Например, если из порта вышли три корабля, а потом сообщили, что погиб один, не обязательно они могли потонуть с равной вероятностью.
	Стоит иметь в виду другие обстоятельства.
	Так зная, что один был особо старый и поломанный, то он погиб с большей вероятностью	
	\item \textit{Следует не только рассматривать доводы, приводящие к утвержденю, но и все те, которые могут привести к противоположному заключению, дабы после должного обсуждения тех и других стало ясно, которые перевешивают.}
	К примеру, если о человеке, который отправился в путешествие, на протяжении двадцати лет ничего не известно, нельз судить, что он мертв. 
	На равне с доводами о том, что путешественники часто гибнут, или он мог умереть от болезни, нужно рассматривать и противоположные доводы: письмо от него пропало или он просто не хотел писать.
	\item \textit{Для суждения о вещах общих достаточны доводы отдаленные и общие; но для суждения о частных вещах следует присоединять также доводы более близкие и спеиальные, если только такие имеются.}
	Так, если спрашивается насколько вероятно, что 20-ти летний человек переживет 60-летнего, то это будут общие суждения, в рассчет которых, кроме этого факта брать нечего.
	Однако, если речь будет идти про конкретных людей, то нужно еще учитывать их состояние здоровья и прочее.
	\item \textit{В обстоятельствах неясных и сомнительных действия должны приостанавливаться, пока не прольется больший свет; но если необходимость действия не терпит отлагательства, из двух исходов нужно всегда избирать тот, который кажется более подходящим <...>, хотя бы ни один таковым на деле не был.}
	Здесь приводится пример следующий. Что если при пожаре остается только один способ спастись -- броситься в окно, то стоит выбирать последнее, так как шансов выжиь больше.
	Хотя очеидно, что этот способ тоже не дает гарантии.
	\item \textit{Что в некотором случае полезно, но ни в каком не вредно, следует предпочитать тому, что никогда не приносит ни пользы, ни вреда.}
	Эта аксиома, как замечает Бернулли, следствие предыдущей.
	\item \textit{Не следует оценивать поступки людей по их результатам.}
	Так например, если некто собирался выбросить тремя бросками на кабиках три шестерки, то он безрассудный, даже если ему это удалось.
	\item \textit{В суждениях наших следует остерегаться, чтобы неприписывать вещам более, чем следует, и не считать самим, а равно и не навязывать другим, за безусловно достоверное, что только вероятнее другого.}
	Здесь поясняется: необходимо, чтобы придаваемая вера соотносилась со степенью достоверности.
	\item \textit{Однако, так как только в редких случаях можно достичь полной достоверномти, то необходимость и обычай требуют, чтобы нравственно лишь достоверное событие считалось безусловно достоверным.}
	Здесь Бернулли рассуждает о том, как было бы неплохо, если бы Правительство установило такие ''пределы'' для нравственной достоверности.
\end{enumerate}
\newpage
\section{Выводы}

\newpage
\begin{thebibliography}{99}
\bibitem{1} Бернулли Я. \textit{О законе больших чисел.} М., Наука. 1986.
\bibitem{2} https://ru.wikipedia.org/wiki/Бернулли,\_Якоб
\bibitem{3} Хрестоматия по истории математики. \textit{Под ред. Юшкевича А.П.} М., Просвещение. Т. 1–2. 1976–1977.

\end{thebibliography}

\end{document}