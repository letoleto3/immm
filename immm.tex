\documentclass[12pt]{article}
\usepackage[utf8]{inputenc}
\usepackage[T2A]{fontenc}
\usepackage[russian]{babel}
\usepackage{hyperref}
\usepackage{cite,enumerate,float,indentfirst}

\usepackage{amsthm}
\usepackage{amsbsy}
\usepackage{amsmath}
\usepackage{amssymb}
\usepackage{amsfonts}
\usepackage{mathtext}
\usepackage{float}

\usepackage{array}
\usepackage{longtable}

\usepackage[pdftex]{graphicx}
\usepackage{graphics}
\usepackage{xcolor}
\usepackage{tikz}
\usepackage{verbatim}

\newtheorem{example}{Пример}[section]
\newtheorem{definition}{Определение}[section]
\newtheorem{proposition}{Предложение}[section]
\newtheorem{theorem}{Теорема}[section]
\newtheorem{corollary}{Следствие}[section]
\newtheorem{lemma}{Лемма}[section]

\DeclareMathOperator{\tr}{tr}

\newcommand{\dif}[2]{\frac{\partial #1}{\partial #2}}

%\parindent = 30pt
%\hoffset = 0pt
%\voffset = 0pt
%\oddsidemargin = 72pt
%\topmargin = 18pt
%\headheight = 12pt
%\headsep = 25pt
%\textheight = 568pt
%\textwidth = 360pt
%\marginparsep = 10pt
%\marginparwidth = 103pt
%\paperwidth = 597pt
%\paperheight = 845pt


\begin{document}

\renewcommand{\contentsname}{Содержание}

\begin{titlepage} \newpage 
	\begin{center} МОСКОВСКИЙ ГОСУДАРСТВЕННЫЙ УНИВЕРСИТЕТ ИМ. М.В.ЛОМОНОСОВА\\ \end{center} 
	\vspace{3em} 
	\begin{center} \Large Механико-математический факультет \\ \end{center}
	\vspace{5em} 
	\begin{center} 
		\textsc{
			\textit{Аннотация к книге} \linebreak
			\textbf{
				Я.Бернулли \linebreak
				\large{О законе больших чисел}
			}
		}
	\end{center}
	\vspace{6em} \newbox{\lbox} \savebox{\lbox}{\hbox{Нагорных Яна}} 
	\newlength{\maxl} \setlength{\maxl}{\wd\lbox} \hfill\parbox{11cm}
	{ \hspace*{5cm}\hspace*{-5cm}Студент:\hfill\hbox to\maxl{Нагорных Я.В.}\\
	\hspace*{5cm}\hspace*{-5cm}Преподаватель:\hfill\hbox to\maxl{Подколзина М.А.}\\ \\
	\hspace*{5cm}\hspace*{-5cm}Группа:\hfill\hbox to\maxl{410}\\ } 
	\vspace{\fill}
	\begin{center} Москва \\ 2018 \end{center} 
\end{titlepage}

\tableofcontents

\newpage
\section{Краткая биография}
Якоб Бернулли, 6 января 1655, Базель, — 16 августа 1705, там же) — швейцарский математик. Один из основателей теории вероятностей и математического анализа. Старший брат Иоганна Бернулли, совместно с ним положил начало вариационному исчислению. Доказал частный случай закона больших чисел — теорему Бернулли[6]. Профессор математики Базельского университета (с 1687 года)[6]. Иностранный член Парижской академии наук (1699) и Берлинской академии наук (1702).


Якоб Бернулли родился 6 января 1655 года в швейцарском Базеле.
Младший брат Якоба Иоганн также стал математиком.
Его отец Николай был преуспевающим фармацевтом. 

По его же желанию Якоб пошел учиться богословию в Базельский университет, однако увлёкся математикой, которую изучил самостоятельно. 
В университете Якоб выучил французский, итальянский, английский, латинский и греческий языки.
В 1671 году он получил учёную степень магистра философии.
Позднее Бернулли много путешествовал по Европе, что дало ему знакомства с Гюйгенсом, Гуком и Бойлем.

В 1683 году вернулся в Базель после путешествий и вскоре женился на Юдит Штупанус. Позднее у них родились сын и дочь.

В 1683 году начал читать лекции по физике в Базельском университете. 
С 1687 года избран профессором математики в этом университете и работал там до конца жизни.
В то же время он начал изучать труды Лейбница по анализу и с энтузиазмом начал освоение нового исчисления. Также он привлек в изучение анализа своего младшего брата Иоганна.
Позднее Лейбниц откликнулся на письмо Якоба, и их взаимная переписка обогатила зарождающийся анализ.

В 1699 году Якоб Бернулли стал членом Парижской академии наук, а в 1702 году -- Берлинской академии наук.

Первое триумфальное выступление молодого математика относится к 1690 году. Якоб решает задачу Лейбница о форме кривой, по которой тяжелая точка опускается за равные промежутки времени на равные вертикальные отрезки. Лейбниц и Гюйгенс уже установили, что это полукубическая парабола, но лишь Якоб Бернулли опубликовал доказательство средствами нового анализа, выведя и проинтегрировав дифференциальное уравнение. При этом впервые появился в печати термин <<интеграл>>.

Якоб Бернулли внёс огромный вклад в развитие аналитической геометрии и зарождение вариационного исчисления. Его именем названа лемниската Бернулли. Он исследовал также циклоиду, цепную линию, и особенно логарифмическую спираль.

Якобу Бернулли принадлежат значительные достижения в теории рядов, дифференциальном исчислении, теории вероятностей и теории чисел, где его именем названы <<числа Бернулли>>.

Он изучил теорию вероятностей по книге Гюйгенса <<О расчётах в азартной игре>>, в которой ещё не было определения и понятия вероятности. 
Якоб Бернулли ввёл значительную часть современных понятий теории вероятностей и сформулировал первый вариант закона больших чисел. 
Он подготовил монографию в этой области, однако издана она была посмертно в 1713 году его братом Николаем, под названием <<Искусство предположений>> (Ars conjectandi). 
Это содержательный трактат по теории вероятностей, статистике и их практическому применению, итог комбинаторики и теории вероятностей XVII века.
Имя Якоба носит важное в комбинаторике распределение Бернулли.

Якоб Бернулли издал также работы по различным вопросам арифметики, алгебры, геометрии и физики. 

В 1692 году у Якоба Бернулли обнаружились первые признаки туберкулёза, от которого он и скончался 16 августа 1705 года. Похоронен в Базельском соборе. Согласно его пожеланию, на надгробии была изображена логарифмическая спираль.

\newpage
\section{Описание книги}


\newpage
\section{Выводы}

\newpage
\begin{thebibliography}{99}
\bibitem{1} Бернулли Я. \textit{О законе больших чисел.} М., Наука. 1986.
\bibitem{2} https://ru.wikipedia.org/wiki/Бернулли,\_Якоб

\end{thebibliography}

\end{document}